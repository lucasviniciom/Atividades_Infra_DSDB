\documentclass{article}
\usepackage[utf8]{inputenc}

\title{Prática em Redes, Internet e a Web}
\author{Lucas Vinicio Maldonado}
\date{Novembro 2021}

\begin{document}

\maketitle

\section{Endereço do repositório Github:}
https://github.com/lucasviniciom/Atividades_Infra_DSDB

\section{Quais as diferenças na estrutura da rede IPÊ de 2016 (slide 8) para 2020/2021?}
Além de existirem muitas novas conexões entre cidades, a capacidade aumentou significativamente. Existem diversas conexões com capacidade de 100Gbps no Nordeste agora, uma conexão de 200Gbps entre São Paulo e Fortaleza, além de mais uma conexão internacional que não existia.

\section{Qual a diferença entre Web e Internet?}
A web é um subdominio dentro de toda a infraestrutura da Internet. Consiste no compartilhamento de arquivos pelo protocolo HTTP, no qual a parte mais conhecida são as paginas web, hospedadas em servidores espalhados em inúmeras regiões.
A internet é uma estrutura muito maior, na qual podem ser usados inúmeros outros protocolos com outras finalidades e níveis de segurança.

\section{Quais órgãos administram o ponto br (.br) para além do slide 17?}
Outros orgãos que administram o .br:
CEWEB.br; CETIC.br; CEPTRO.br; W3C.br; NTP.br;


\section{Quais características do protocolo HTTP descritas na RFC você já conhecia?}
Já havia visto algo sobre as versões do HTTP, alguns dos erros mais comuns como 404 e os métodos como GET, POST e PUT.


\section{Qual o motivo de haver 2 chaves diferentes na figura do slide 32?}

Ilustra a criptografia de ponta a ponta. A chave pública sera usada para criptografar o arquivo e somente o detentor da privada, a outra ponta que é o destinatário, vai poder descriptografar. Um exemplo é o SSH.

\end{document}
