\documentclass{article}
\usepackage[utf8]{inputenc}

\title{Prática em Redes, Internet e a Web}
\author{Lucas Vinicio Maldonado}
\date{November 2021}

\begin{document}

\maketitle
\section{Endereço do repositorio Github}
\url{https://github.com/lucasviniciom/Atividades\_Infra\_DSDB}

\section{Harvard Art Museum API}
A API do Museu de Artes de Harvard permite explorar a coleção do Museu e integrar seus dados. Nesta aplicação básica, é escolhido uma imagem aleatoriamente e exibida ao usuário. 

\section{Descrição da aplicação}
É primeiramente escolhido um número aleatório dentro de um range no qual a maioria dos números possui uma obra.
Em seguida é realizado um request no endereço com o método GET, junto com o número aleatório atribuído como chave de uma imagem. Também é enviado a chave para autentificação. 
\\
\\
O endereço DNS é então mapeado e retornado o IP correspondente. Em seguida o metodo GET é propriamente acionado para o IP correspondente. Primeiro é recebido uma resposta se ocorreu com sucesso e em seguida o JSON propriamente com os dados solicitados. Estes dados então são abertos na aplicação e seleciona-se o endereço da imagem, seu nome, a descrição e a data de criação da obra ou da entrada na coleção.
\\
\\
Ao rodar a aplicação, o usuário poderá conferir essas informações no terminal e visualizar a imagem no aplicativo padrão.

\end{document}
