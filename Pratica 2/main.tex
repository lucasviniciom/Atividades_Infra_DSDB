\documentclass{article}
\usepackage[utf8]{inputenc}

\title{Prática em Redes, Internet e a Web}
\author{Lucas Vinicio Maldonado}
\date{November 2021}

\begin{document}

\maketitle
\section{Endereço do repositorio Github}
\text{https://github.com/lucasviniciom/Atividades_Infra_DSDB}

\section{Por que “faltam” camadas no roteador e no switch do slide 7?}
O switch  não direciona através do IP, por isso não tem a camada network. Ambos não possuem App e Transport pois não são feitos para navegação, apenas para redirecionar as mensagens.

\section{Por que seu roteador Wi-Fi não é um roteador “de verdade”?}
O real roteador fica com o provedor de internet. É ele quem vai trazar a rota e direcionar um request HTTP, por exemplo, para o servidor certo. O roteador Wifi apenas modifica o sinal de internet para ser acessível por ondas eletromagnéticas no ar.

\section{Qual a porta padrão dos seguintes protocolos: DHCP, HTTPS e POP3?}
DHCP - 67; 
HTTPS - 443;
POP3 - 110


\section{Ainda há endereços IPv4 disponíveis no Brasil? Quando esgotaram ou quando esgotarão?}
Segundo a fonte a seguir, terminaram ano passado os endereços disponiveis. Há porem ainda uma fila para quem tem interesse, caso venham a se tornar disponíveis.
https://ipv6.br/post/fim-do-ipv4/

\section{Qual o ip local da sua máquina? E da macalan?}
Local: 192.168.0.47

Macalan: 200.17.202.6

\section{Qual a rota padrão da sua máquina? E da macalan?}
Local: 192.168.0.1

Rota padrão: 200.17.202.62

\section{Qual o caminho (route) mais comum entre sua máquina e a macalan?}

Local pra Macalan:
Tracing route to 200.17.202.62 over a maximum of 30 hops
  1     4 ms     2 ms     3 ms  192.168.0.1
  
Da Macalan pra Local:
traceroute to 192.168.0.47 (192.168.0.47), 30 hops max, 60 byte packets
 1  _gateway (200.17.202.62)  0.099 ms !H  0.067 ms !H *
\section{A partir de diferentes máquinas, o caminho (route) até a macalan muda?}

Não, pois o caminho depende da localização física e do provedor de internet.

\end{document}
